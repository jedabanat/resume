% Jed Abanat Resume

\documentclass[10pt,letterpaper]{article}
\usepackage[letterpaper,margin=0.5in]{geometry}
\usepackage[utf8]{inputenc}
\usepackage{mdwlist}
\usepackage[T1]{fontenc}
\usepackage{textcomp}
\usepackage[none]{hyphenat}
\usepackage{tgpagella}
\usepackage[hidelinks]{hyperref}
\usepackage[document]{ragged2e} % justifying
\usepackage{multicol}
\pagestyle{empty}
\setlength{\tabcolsep}{0em}

% indentsection style, used for sections that aren't already in lists
% that need indentation to the level of all text in the document
\newenvironment{indentsection}[1]%
{\begin{list}{}%
	{\setlength{\leftmargin}{#1}}%
	\item[]%
}
{\end{list}}

% opposite of above; bump a section back toward the left margin
\newenvironment{unindentsection}[1]%
{\begin{list}{}%
	{\setlength{\leftmargin}{-0.5#1}}%
	\item[]%
}
{\end{list}}

% format two pieces of text, one left aligned and one right aligned
\newcommand{\headerrow}[2]
{\begin{tabular*}{\linewidth}{l@{\extracolsep{\fill}}r}
	#1 &
	#2 \\
\end{tabular*}}

% make "C++" look pretty when used in text by touching up the plus signs
\newcommand{\CPP}
{C\nolinebreak[4]\hspace{-.05em}\raisebox{.22ex}{\footnotesize\bf ++}}

\tolerance=1
\emergencystretch=\maxdimen
\hyphenpenalty=10000
\hbadness=10000
\setlength{\parindent}{0pt}

% and the actual content starts here
\begin{document}

\begin{center}
    {\Huge \textbf{Jed Abanat}} \\
        \vspace{4pt}
        0401 465 246 $|$ 
        \href{mailto:jedabanat@gmail.com}{\underline{jedabanat@gmail.com}} $|$ 
        \href{https://linkedin.com/in/jedabanat}{\underline{linkedin.com/in/jedabanat}}
\end{center}
\vspace{1em}
I am an experienced software engineer proficient in building and maintaining Windows desktop applications in C\# with a strong understanding of designing \& implementing signal processing algorithms. I’ve worked autonomously as a sole developer for most of my career, leading development initiatives from conception to global deployment. I actively question the requirements and collaborate cross-functionally to ensure the best solution is developed. As a  passionate advocate for automation, I strive to streamline processes within both the development workflow and the product itself.

\subsection*{Work Experience}
\vspace{-0.25em}
\hrule
\vspace{1em}
\parskip=0.1em

\headerrow
    {\textbf{Career Break / Travel}}
    {\textit{July 2023 - Present}}
    \begin{itemize*}
    \vspace{-0.5em}
    \item Resigned to travel and relocate back to Australia.  
        \end{itemize*}

\headerrow
    {\textbf{Fugro USA Marine}}
    {\textbf{Houston, TX, USA}}
	\headerrow
		{\emph{Innovation (Software) Engineer}}
		{\emph{Sep 2019 - July 2023}} \\
  
    \begin{itemize*}
    \vspace{-0.5em}
    \item Technical lead of a software suite which processed airborne lidar bathymetry data. Wrote 70\% of the code and reviewed the remaining 30\%. 
    \item \justifying{Developed a novel bathymetric lidar processing solution with the ML team which replaced traditional signal processing algorithms with an image recognition approach. Resulted in 95\% less noise \& improved accuracy of final product by 30\%. Filed a patent for this invention - \textit{‘Machine Learning Processing of Airborne Lidar Bathymetric Waveform Data’}}
    \item Refactored original MATLAB processing app into a .NET console app in C\#, following SOLID design principles \& implementing Test-Driven Development practices. Increased test coverage from 0\% to 65\%.
    \item Architected and implemented a high-performance data processing pipeline using TPL Dataflow with a multi-process, event-driven approach achieving an 80\% reduction in processing time. Utilized Microsoft ONNX Runtime \& CUDA to run inference on the raw data with NVIDIA GPUs.
    \item Designed user interfaces in WPF using MVVM, working with end-users to achieve an intuitive user experience. The UI displayed the raw data along with various plots and graphs to assist data processors during QC/QA.
    \item Effectively served as Agile Product Owner for 18 months, actively engaging with end-users to gather requirements, prioritise deliverables, and define the development roadmap, ensuring alignment with business objectives. 
    \item Built 2 additional .NET apps to support field staff; an automatic data transfer tool to copy raw data (2TB/flight) \& a QC tool to generate georeferenced images collected during the flight.
    \item Implemented CI/CD workflows in GitHub Actions, reducing release time by 2 hours.
    \item Rotated through two other teams and contributed to separate Windows application development; collaborated with architects to refactor legacy C\# code \& added functionality to encode/decode messages via TCP which allowed operators to remotely control winches. 
    \item Contributed to Visual Studio .NET project configuration templates to standardise all .NET repos in the company.
    \item \justifying{Led on-boarding programs and mentored colleagues, providing technical guidance and demonstrating best-practices.}
    \end{itemize*}

    \headerrow
    {\textbf{Fugro Australia Marine}}
    {\textbf{Adelaide, Australia}}

	\headerrow
		{\emph{Airborne Software Engineer}}
		{\emph{Mar 2018 - Sep 2019}} \\
    \begin{itemize*}
    \vspace{-0.5em}
\item Maintained and improved an acquired MATLAB code-base as the sole developer, working autonomously to fix bugs and add additional features requested by data processors.
\item Researched, tested and implemented signal processing \& noise filtering algorithms to meet client requirements. 
\item Travelled to Houston regularly to meet directly with end-users and external clients to gather requirements and collect app feedback. Provided hands-on field support and gained additional context to proactively identify \& address bottlenecks in the processing workflow.
\item Led fortnightly planning and demo meetings with end-users to ensure alignment on key priorities and to collect feedback on new features \& bug fixes, resulting in improved user satisfaction.
\item Automated the deployment of the MATLAB code using Ruby scripts to compile (MATLAB Compiler) and generate an installer (WiX Toolkit).
\item Developed comprehensive test procedure documents to compare the performance and accuracy of Fugro’s lidar sensors. Designed a rubric and scoring system for fair comparison between the sensors and provided recommendations to management for future development efforts.
\item Designed mounting plates in AutoCAD to install lidar sensors in aircraft, adhering to all Civil Aviation Safety Authority (CASA) standard while ensuring mounting plate was adaptable to multiple aircraft.
    \end{itemize*}

\newpage
\vspace{.5em}
\headerrow
{\emph{Engineering Intern}}
{\emph{Jan 2018 - Mar 2018}} \\
 \begin{itemize*}
\vspace{-0.5em}
\item Independently conducted thorough analysis of an acquired code-base in MATLAB (\textasciitilde 8000 lines) to identify and understand each component and function of the process.  
\item Produced comprehensive technical documentation and detailed flowchart diagrams explaining the code to global stakeholders, resulting in improved cross-functional collaboration between the Adelaide \& Houston offices. 
\item Designed a UI for the code using MATLAB GUIDE, working with end-users to design an intuitive \& interactive tool.

\end{itemize*}    

\subsection*{Technical Skills}
\vspace{-0.25em}
\hrule
\vspace{1em}
\textbf{Languages:} C\#, MATLAB, Ruby, Python\\
\textbf{Frameworks:} .NET (Core \& Framework), WPF, Blazor \\
\textbf{Developer Tools:} Git, Jira, TestRail, WiX Toolkit, Inno Setup 

% EDUCATION %
\subsection*{Education}
\hrule
\vspace{1em}
{\textbf{The University of Adelaide}} \\
\headerrow
    {\emph{Bachelor of Engineering (Mechatronics) (Honours)}}
    {\emph{2014 - 2018}}

\end{document}
